% !TEX root = ../my-thesis.tex
%
\pdfbookmark[0]{Abstract}{Abstract}
\addchap*{Abstract}
\label{sec:abstract}

This thesis presents a comprehensive study on developing a machine learning model to predict mobility patterns in various weather conditions and group structures. The primary focus of this project is to assess the effectiveness of machine learning models in forecasting mobility behavior, considering different weather scenarios, social contexts, and route characteristics. Additionally, the research aims to identify the key factors influencing mobility behavior and explore the potential application of machine learning models in personalized mobility prediction and sustainable transportation planning.

To achieve these objectives, a range of machine learning models will be evaluated for their accuracy and precision in forecasting mobility patterns. The research will investigate how different models perform in predicting mobility behavior under varying weather conditions and social contexts. Moreover, an in-depth analysis will be conducted to identify the factors that significantly impact the accuracy of these models.

By examining the accuracy and precision of various machine learning models, this research will provide valuable insights into their effectiveness in mobility forecasting. Furthermore, it will uncover the factors that play a crucial role in influencing the accuracy of these models. The findings of this study will contribute to the advancement of machine learning applications in transportation planning and assist in developing personalized mobility prediction systems for sustainable transportation.

Keywords: machine learning, mobility patterns, weather conditions, group structures, accuracy, precision, personalized mobility prediction, sustainable transportation planning

\vspace*{20mm}

{\usekomafont{chapter}Astratto}
\label{sec:abstract-diff}

Questa tesi presenta uno studio approfondito per lo sviluppo di un modello di apprendimento automatico per prevedere i modelli di mobilità in diverse condizioni meteorologiche e strutture di gruppo. Il focus principale di questo progetto è valutare l'efficacia dei modelli di apprendimento automatico nella previsione del comportamento di mobilità, considerando diversi scenari meteorologici, contesti sociali e caratteristiche del percorso. Inoltre, la ricerca mira a identificare i fattori che influenzano il comportamento di mobilità ed esplorare la possibilità di utilizzare i modelli di apprendimento automatico per la previsione della mobilità personalizzata e la pianificazione del trasporto sostenibile.

Per raggiungere questi obiettivi, verranno valutati diversi modelli di apprendimento automatico per la loro accuratezza e precisione nella previsione dei modelli di mobilità. La ricerca indagherà su come diversi modelli si comportano nella previsione del comportamento di mobilità in diverse condizioni meteorologiche e contesti sociali. Inoltre, verrà condotta un'analisi approfondita per identificare i fattori che influenzano in modo significativo l'accuratezza di questi modelli.

Attraverso l'esame dell'accuratezza e della precisione dei vari modelli di apprendimento automatico, questa ricerca fornirà preziose intuizioni sulla loro efficacia nella previsione della mobilità. Inoltre, svelerà i fattori che giocano un ruolo cruciale nell'influenzare l'accuratezza di questi modelli. I risultati di questo studio contribuiranno all'avanzamento delle applicazioni di apprendimento automatico nella pianificazione dei trasporti e aiuteranno nello sviluppo di sistemi di previsione della mobilità personalizzata per il trasporto sostenibile.

Parole chiave: apprendimento automatico, modelli di mobilità, condizioni meteorologiche, strutture di gruppo, accuratezza, precisione, previsione della mobilità personalizzata, pianificazione del trasporto sostenibile.





